\documentclass{article}
\usepackage[utf8]{inputenc}

\usepackage[margin=1in]{geometry}

\title{Discrete Math: Problem Set #1}
\author{Jackie Luo}
\date{February 5, 2015}

\begin{document}

\maketitle

\section{1.1, 12f}
You have the flu and miss the final examination, or you do not miss the final examination and you pass the course.

\section{1.1, 14f}
p \iff q \lor r

\section{1.1, 28c}
Converse: If it is necessary that I sleep until noon, then I stayed up late.
\newline
Contrapositive: If it is not necessary that I sleep until noon, then I did not stay up late.
\newline
Inverse: If I did not stay up late, then it is not necessary that I sleep until noon.

\section{1.1, 32d}
\begin{table}[h]
\centering
\begin{tabular}{|l|l|l|l|l|}
\hline
p & q & p \wedge q & p \vee q & p \wedge q \to p \vee q \\ \hline
T & T & T & T & T \\ \hline
T & F & F & T & T \\ \hline
F & T & F & T & T \\ \hline
F & F & F & F & T \\ \hline
\end{tabular}
\end{table}

\section{1.1, 36}

\subsection{e}
\begin{table}[h]
\centering
\begin{tabular}{|l|l|l|l|l|l|}
\hline
p & q & r & p \vee q & \neg r & p \vee q \wedge \neg r \\ \hline
T & T & T & T & F & F \\ \hline
T & T & F & T & T & T \\ \hline
T & F & T & T & F & F \\ \hline
T & F & F & T & T & T \\ \hline
F & T & T & T & F & F \\ \hline
F & T & F & T & T & T \\ \hline
F & F & T & F & F & F \\ \hline
F & F & F & F & T & F \\ \hline
\end{tabular}
\end{table}

\subsection{f}
\begin{table}[h]
\centering
\begin{tabular}{|l|l|l|l|l|l|}
\hline
p & q & r & p \wedge q & \neg r & p \wedge q \vee \neg r \\ \hline
T & T & T & T & F & T \\ \hline
T & T & F & T & T & T \\ \hline
T & F & T & F & F & F \\ \hline
T & F & F & F & T & T \\ \hline
F & T & T & F & F & F \\ \hline
F & T & F & F & T & T \\ \hline
F & F & T & F & F & F \\ \hline
F & F & F & F & T & F \\ \hline
\end{tabular}
\end{table}

\section{1.3, 4b}
\begin{table}[h]
\centering
\begin{tabular}{|l|l|l|l|l|l|l|l|}
\hline
p & q & r & p \wedge q & (p \wedge q) \wedge r & q \wedge r & p \wedge (q \wedge r) & (p \wedge q) \wedge r \equiv p \wedge (q \wedge r) \\ \hline
T & T & T & T & T & T & T & T \\ \hline
T & T & F & T & F & F & F & T \\ \hline
T & F & T & F & F & F & F & T \\ \hline
T & F & F & F & F & F & F & T \\ \hline
F & T & T & F & F & T & F & T \\ \hline
F & T & F & F & F & F & F & T \\ \hline
F & F & T & F & F & F & F & T \\ \hline
F & F & F & F & F & F & F & T \\ \hline
\end{tabular}
\end{table}

\section{1.3, 24}
\begin{table}[h]
\centering
\begin{tabular}{|l|l|l|l|l|l|l|l|l|}
\hline
p & q & r & p \to q & p \to r & (p \to q) \vee (p \to r) & q \vee r & p \to (q \vee r) & ((p \to q) \vee (p \to r)) \equiv (p \to (q \vee r)) \\ \hline
T & T & T & T & T & T & T & T & T \\ \hline
T & T & F & T & F & T & T & T & T \\ \hline
T & F & T & F & T & T & T & T & T \\ \hline
T & F & F & F & F & F & F & F & T \\ \hline
F & T & T & T & T & T & T & T & T \\ \hline
F & T & F & T & T & T & T & T & T \\ \hline
F & F & T & T & T & T & T & T & T \\ \hline
F & F & F & T & T & T & F & T & T \\ \hline
\end{tabular}
\end{table}

\section{1.4, 16d}
\forall x(x^2 \neq x)$ is false because $x^2 \neq x$ is not true for all real numbers. Two cases for which it is false are $x = 0$ and $x = 1$, as $0^2 = 0$ and $1^2 = 1.

\section{1.4, 50}
\forall P(x) \vee \forall Q(x)$ states that either $P(x)$ is true for every $x$ or $Q(x)$ is true for every $x$, meaning that only one of the statements is true for every $x$ in the domain. On the other hand, $\forall(P(x) \vee Q(x))$ states that either $P(x)$ or $Q(x)$ is true for every $x$. In this case, within the domain, both statements can be true, and one must be true for every $x$ in the domain.

\section{1.5, 10f}
\neg \exists x(F(x,Fred) \wedge F(x, Jerry))

\section{1.5, 38c}
There is not a student in this class who has taken every mathematics course offered at this school. \neg \exists x(P(x))$, where $x$ is a student, the domain is the students in the class, and $P(x)$ is the statement that $x$ has taken every mathematics course offered at the school.$

\section{1.6, 10c}
If all insects have six legs and dragonflies are insects, then dragonflies have six legs. If all insects have six legs and spiders do not have six legs, then spiders are not insects. If dragonflies are insects and spiders eat dragonflies, then spiders eat some insects.

\section{1.6, 34}
p: "Logic is difficult."
\newline
q: "Many students like logic."
\newline
r: "Mathematics is easy."
\newline
1. p \vee \neg q
\newline
2. r \iff \neg p
\newline
\newline
$It is valid to say that mathematics is not easy if many students like logic because logic is difficult if and only if mathematics is not easy, and many students like logic if logic is difficult. It is not valid to conclude that mathematics is not easy or logic is difficult because if mathematics is not easy, then logic is difficult, meaning that the "or" compares two equivalent statements.$

\section{1.7, 42}
If n = 3:
\newline
i. n^2$ is odd is true, as $3^2 = 9$, an odd number$
\newline
$ii. $1 - n$ is even is true, as $1 - 3 = -2$, an even negative number$
\newline
$iii. $n^3$ is odd is true, as $3^3 = 27$, an odd number$
\newline
$iv. $n^2 + 1$ is even is true, as $3^2 + 1 = 10$, an even number$
\newline
True \equiv True \equiv True \equiv True

\section{1.8, 26}
Let $p$ be the proposition that, given five ones and four zeroes arranged in a circle, if you place a zero between any two equal bits and a one between any two unequal bits, then erase the original bits, the remaining bits will never be nine zeroes. Suppose that \neg p$ is true. If there are nine zeroes, then the previous iteration had all equal bits, which must have been nine ones. If there were nine ones, then the previous iteration must have been a circle of unequal bits, with ones and zeroes (no adjacent equal bits). However, in a circle of nine bits, it is impossible for every adjacent bit to be unequal; there will always be at least one pair of equal bits (two ones or two zeroes) next to each other. As a result, it is impossible for the circle to consist of nine zeroes, and $p$ must be true.$

\section{1.8, 30}
2x^2 + 5y^2 = 14
x$ and $y$ must be integers with absolute values between 0 and 3, as the integers squared will always be positive, and neither $x^2$ or $y^2$ can be greater than 14. The largest perfect square less than that is 9, and its square root is 3. We can narrow the numbers further; $x$ must be between 0 and 2 because $2x^2 = 14$ yields $\sqrt{7}$, meaning that 3 ($\sqrt{9}$) must be too large. Similarly, with $5y^2 = 14$, $y = \sqrt{2.8}$, so $y$ would have to be either 0 or 1 (2 is $\sqrt{4}$, again too large).

\begin{table}[h]
\centering
\begin{tabular}{lllll}
 & x &  &  &  \\ \cline{2-5} 
\multicolumn{1}{l|}{y} & \multicolumn{1}{l|}{} & \multicolumn{1}{l|}{0} & \multicolumn{1}{l|}{1} & \multicolumn{1}{l|}{2} \\ \cline{2-5} 
\multicolumn{1}{l|}{} & \multicolumn{1}{l|}{0} & \multicolumn{1}{l|}{0} & \multicolumn{1}{l|}{2} & \multicolumn{1}{l|}{8} \\ \cline{2-5} 
\multicolumn{1}{l|}{} & \multicolumn{1}{l|}{1} & \multicolumn{1}{l|}{5} & \multicolumn{1}{l|}{7} & \multicolumn{1}{l|}{13} \\ \cline{2-5} 
\end{tabular}
\end{table}

Testing all of the possible options, we can conclude that there are no solutions in integers $x$ and $y$ for the equation.

\end{document}
