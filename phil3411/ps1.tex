\documentclass{article}
\usepackage[utf8]{inputenc}

\usepackage[margin=1in]{geometry}

\title{Symbolic Logic: Homework #2}
\author{Jackie Luo}
\date{February 10, 2015}

\begin{document}

\maketitle

\section{2.2}
\subsection{$\neg((\neg A) \wedge \neg(B \wedge A))$}

\begin{table}[h]
\centering
\begin{tabular}{|l|l|l|l|l|}
\hline
A & B & B \wedge A & (\neg A) \wedge \neg(B \wedge A) & \neg((\neg A) \wedge \neg(B \wedge A)) \\ \hline
T & T & T & F & T \\ \hline
T & F & F & F & T \\ \hline
F & T & F & T & F \\ \hline
F & F & F & T & F \\ \hline
\end{tabular}
\end{table}

\newline
\subsection{$(\neg((\neg A) \wedge \neg(B \wedge C))) \wedge (A \wedge B)$}

\begin{table}[h]
\centering
\begin{tabular}{|l|l|l|l|l|l|l|}
\hline
A & B & C & B \wedge C & (\neg A) \wedge \neg(B \wedge C) & A \wedge B & \neg((\neg A) \wedge \neg(B \wedge C)) \wedge (A \wedge B) \\ \hline
T & T & T & T & F & T & T \\ \hline
T & T & F & F & F & T & T \\ \hline
T & F & T & F & F & F & F \\ \hline
T & F & F & F & F & F & F \\ \hline
F & T & T & T & F & F & F \\ \hline
F & T & F & F & T & F & F \\ \hline
F & F & T & F & T & F & F \\ \hline
F & F & F & F & T & F & F \\ \hline
\end{tabular}
\end{table}

\newline
\subsection{$\neg(A \wedge ((\neg A) \wedge (\neg B)))$}

\begin{table}[h]
\centering
\begin{tabular}{|l|l|l|l|}
\hline
A & B & (\neg A) \wedge (\neg B) & \neg(A \wedge ((\neg A) \wedge (\neg B))) \\ \hline
T & T & F & T \\ \hline
T & F & F & T \\ \hline
F & T & F & T \\ \hline
F & F & T & T \\ \hline
\end{tabular}
\end{table}

\newpage
\subsection{$(\neg(A \wedge ((\neg A) \wedge (\neg C)))) \wedge (\neg (\neg B))$}

\begin{table}[h]
\centering
\begin{tabular}{|l|l|l|l|l|l|}
\hline
A & B & C & (\neg A) \wedge (\neg C) & A \wedge ((\neg A) \wedge (\neg C)) & \neg(A \wedge ((\neg A) \wedge (\neg C))) \wedge \neg (\neg B) \\ \hline
T & T & T & F & F & T \\ \hline
T & T & F & F & F & T \\ \hline
T & F & T & F & F & T \\ \hline
T & F & F & F & F & T \\ \hline
F & T & T & F & F & F \\ \hline
F & T & F & T & F & F \\ \hline
F & F & T & F & F & F \\ \hline
F & F & F & T & F & F \\ \hline
\end{tabular}
\end{table}

\newline
\subsection{$C \wedge (\neg(C \wedge \neg(A \wedge C)))$}

\begin{table}[h]
\centering
\begin{tabular}{|l|l|l|l|l|}
\hline
A & C & A \wedge C & C \wedge \neg(A \wedge C) & C \wedge \neg(C \wedge \neg(A \wedge C)) \\ \hline
T & T & T & F & T \\ \hline
T & F & F & F & F \\ \hline
F & T & F & T & F \\ \hline
F & F & F & F & F \\ \hline
\end{tabular}
\end{table}

\newline
\subsection{$A \wedge (\neg(C \wedge ((\neg C) \wedge B)))$}

\begin{table}[h]
\centering
\begin{tabular}{|l|l|l|l|l|l|}
\hline
A & B & C & (\neg C) \wedge B & C \wedge ((\neg C) \wedge B) & A \wedge \neg(C \wedge (\neg C \wedge B)) \\ \hline
T & T & T & F & F & T \\ \hline
T & T & F & T & F & T \\ \hline
T & F & T & F & F & T \\ \hline
T & F & F & T & F & T \\ \hline
F & T & T & F & F & F \\ \hline
F & T & F & F & F & F \\ \hline
F & F & T & F & F & F \\ \hline
F & F & F & F & F & F \\ \hline
\end{tabular}
\end{table}

\section{2.3}
$\neg((\neg A) \wedge \neg(B \wedge A)) \equiv A$
\newline
$(\neg((\neg A) \wedge \neg(B \wedge C))) \wedge (A \wedge B) \equiv A \wedge B$
\newline
$\neg(A \wedge ((\neg A) \wedge (\neg B)))$ can't be simplified.
\newline
$(\neg(A \wedge ((\neg A) \wedge (\neg C)))) \wedge (\neg (\neg B)) \equiv A$
\newline
$C \wedge (\neg(C \wedge \neg(A \wedge C))) \equiv A \wedge C$
\newline
$A \wedge (\neg(C \wedge ((\neg C) \wedge B))) \equiv A$

\newpage
\section{2.4}
1 and 2 ($\neg((\neg A) \wedge \neg(B \wedge A))$ and $(\neg((\neg A) \wedge \neg(B \wedge C))) \wedge (A \wedge B)$)

\begin{table}[h]
\centering
\begin{tabular}{|l|l|l|l|l|l|l|l|l|}
\hline
A & B & C & B \wedge A & \neg((\neg A) \wedge \neg(B \wedge A)) & B \wedge C & (\neg A) \wedge \neg(B \wedge C) & A \wedge B & \neg((\neg A) \wedge \neg(B \wedge C)) \wedge (A \wedge B) \\ \hline
T & T & T & T & T & T & F & T & T \\ \hline
T & T & F & T & T & F & F & T & T \\ \hline
T & F & T & F & T & F & F & F & F \\ \hline
T & F & F & F & T & F & F & F & F \\ \hline
F & T & T & F & F & T & F & F & F \\ \hline
F & T & F & F & F & F & T & F & F \\ \hline
F & F & T & F & F & F & T & F & F \\ \hline
F & F & F & F & F & F & T & F & F \\ \hline
\end{tabular}
\end{table}

3 and 5 ($\neg(A \wedge ((\neg A) \wedge (\neg B)))$ and $C \wedge (\neg(C \wedge \neg(A \wedge C)))$)

\begin{table}[h]
\centering
\begin{tabular}{|l|l|l|l|l|l|l|l|}
\hline
A & B & C & (\neg A) \wedge (\neg B) & \neg(A \wedge ((\neg A) \wedge (\neg B))) & A \wedge C & C \wedge \neg(A \wedge C) & C \wedge \neg(C \wedge \neg(A \wedge C)) \\ \hline
T & T & T & F & T & T & F & T \\ \hline
T & T & F & F & T & F & F & F \\ \hline
T & F & T & F & T & T & F & T \\ \hline
T & F & F & F & T & F & F & F \\ \hline
F & T & T & F & T & F & T & F \\ \hline
F & T & F & F & T & F & F & F \\ \hline
F & F & T & T & T & F & T & F \\ \hline
F & F & F & T & T & F & F & F \\ \hline
\end{tabular}
\end{table}

The other truth tables can be compared as they are because they contain the same variables ($A$, $B$, and $C$). The full table is constructed below, with the truth-values for counterexamples of the non-equivalent statements.

\begin{table}[h]
\centering
\begin{tabular}{|l|l|l|l|l|l|l|}
\hline
 & 1 & 2 & 3 & 4 & 5 & 6 \\ \hline
1 & + & \begin{tabular}[c]{@{}l@{}}A: True\\ B: False\\ C: True\end{tabular} & \begin{tabular}[c]{@{}l@{}}A: False\\ B: False\end{tabular} & + & \begin{tabular}[c]{@{}l@{}}A: True\\ B: True\\ C: False\end{tabular} & + \\ \hline
2 & \begin{tabular}[c]{@{}l@{}}A: True\\ B: False\\ C: True\end{tabular} & + & \begin{tabular}[c]{@{}l@{}}A: True\\ B: False\\ C: True\end{tabular} & \begin{tabular}[c]{@{}l@{}}A: True\\ B: False\\ C: True\end{tabular} & \begin{tabular}[c]{@{}l@{}}A: True\\ B: True\\ C: False\end{tabular} & \begin{tabular}[c]{@{}l@{}}A: True\\ B: False\\ C: True\end{tabular} \\ \hline
3 & \begin{tabular}[c]{@{}l@{}}A: False\\ B: False\end{tabular} & \begin{tabular}[c]{@{}l@{}}A: True\\ B: False\\ C: True\end{tabular} & + & \begin{tabular}[c]{@{}l@{}}A: False\\ B: True\\ C: True\end{tabular} & \begin{tabular}[c]{@{}l@{}}A: True\\ B: True\\ C: False\end{tabular} & \begin{tabular}[c]{@{}l@{}}A: False\\ B: True\\ C: True\end{tabular} \\ \hline
4 & + & \begin{tabular}[c]{@{}l@{}}A: True\\ B: False\\ C: True\end{tabular} & \begin{tabular}[c]{@{}l@{}}A: False\\ B: True\\ C: True\end{tabular} & + & \begin{tabular}[c]{@{}l@{}}A: True\\ B: True\\ C: False\end{tabular} & + \\ \hline
5 & \begin{tabular}[c]{@{}l@{}}A: True\\ B: True\\ C: False\end{tabular} & \begin{tabular}[c]{@{}l@{}}A: True\\ B: True\\ C: False\end{tabular} & \begin{tabular}[c]{@{}l@{}}A: True\\ B: True\\ C: False\end{tabular} & \begin{tabular}[c]{@{}l@{}}A: True\\ B: True\\ C: False\end{tabular} & + & \begin{tabular}[c]{@{}l@{}}A: True\\ B: True\\ C: False\end{tabular} \\ \hline
6 & + & \begin{tabular}[c]{@{}l@{}}A: True\\ B: False\\ C: True\end{tabular} & \begin{tabular}[c]{@{}l@{}}A: False\\ B: True\\ C: True\end{tabular} & + & \begin{tabular}[c]{@{}l@{}}A: True\\ B: True\\ C: False\end{tabular} & + \\ \hline
\end{tabular}
\end{table}

\newpage
\section{2.5}
\begin{table}[h]
\centering
\begin{tabular}{|l|l|l|l|l|l|}
\hline
A & B & A \wedge B & A \vee^x B & (A \wedge B) \vee^x (A \vee^x B) & A \vee B \\ \hline
T & T & T & F & T & T \\ \hline
T & F & F & T & T & T \\ \hline
F & T & F & T & T & T \\ \hline
F & F & F & F & F & F \\ \hline
\end{tabular}
\end{table}
$(A \wedge B) \vee^x (A \vee^x B)$ is equivalent to $A \vee B$ and uses only "and" and "exclusive or" symbols.

\end{document}