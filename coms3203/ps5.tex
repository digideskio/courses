\documentclass{article}
\usepackage[utf8]{inputenc}
\usepackage[margin=1in]{geometry}

\title{Discrete Math: Problem Set 5}
\author{Jackie Luo}
\date{April 23, 2015}

\begin{document}
\maketitle

\section{7.1, 36}
There are 36 pairings of the two dice ($6 \times 6$), and there are 5 ways to pair the dice to get a sum of 8: (2, 6), (3, 5), (4, 4), (5, 3), and (6, 2). Thus, the probability of rolling an 8 with two dice is 5/36, or 0.139. For three dice, the total number of pairings is 216 ($6^3$). The number of ways to get a sum of 8 is C(7, 5), or 21, which makes the total probability 21/216, or 0.097. It's more likely to roll an 8 with two dice than three.

\section{7.2, 10c}
The letters "a" and "z" can be next to each other in 2 different ways. The other letters in the permutations can be ordered in any way, so the probability of the event is $(2 \times 25!)/26!$, or 0.077.

\section{7.2, 10f}
The letter "z" can precede the letters "a" and "b" in C(26, 3) ways, or 2600 ways. We can multiply that by 2 because "a" and "b" can be in different positions, and there are 23! ways to arrange the other letters. Thus, the probability is $(2600 \times 2 \times 23!)/26!$, or 1/3.

\section{7.2, 34}
a. The probability of no successes is the probability that all of the trials will be failures, so it's $(1 - p)^n$.
\newline
b. The probability that there will be at least one success is the opposite of the probability that all of the trials will be failures, so the probability is $1 - (1 - p)^n$.
\newline
c. The probability of at most one success is the sum of the probability of no successes and the probability of one success. The probability of no successes is, as mentioned earlier, $(1 - p)^n$, and the probability of one success is $n \times p \times (1 - p)^{n - 1}$. The sum is $(1 - p)^n + n \times p \times (1 - p)^{n - 1}$.
\newline
d. The probability of at least two successes is the opposite of the probability that at most one of the trials will be a success, so the probability is $1 - (1 - p)^n + n \times p \times (1 - p)^{n - 1}$.

\section{8.1, 12}
a. If $a_n$ is the number of ways to climb $n$ stairs if the person is climbing one, two, or three stairs at a time, then the recurrence relation is $a_n = a_{n - 1} + a_{n - 2} + a_{n - 3}$ for $x \geq 3$.
\newline
b. The initial conditions are $a_0 = 1$, $a_1 = 1$, and $a_2 = 2$.
\newline
c. $a_8 = 81$

\section{8.1, 28}
For $n \geq 5$:
\newline
$f_n = f_{n - 1} + f_{n - 2}
\newline
= (f_{n - 2} + f_{n - 3}) + (f_{n - 3} + f_{n - 4})
\newline
= (f_{n - 3} + f_{n - 4}) + 2f_{n - 3} + f_{n - 4}
\newline
= 3f_{n - 3} + 2f_{n - 4}
\newline
= (3f_{n - 4} + 3f_{n - 5}) + 2f_{n - 4}
\newline
= 5f_{n - 4} + 3f_{n - 5}
\newline
f_5 = f_3 + f_4 = 5
\newline
f_{5(n + 1)} = f_{5n + 5}
\newline
= 5f_{5(n + 1) - 4} + 3f_{5(n + 1) - 5}
\newline
= 5f_{5n + 1} + 3f_{5n}
\newline
= 5f_{5n + 1} + 15k$
\newline
Both coefficients of the components in the expression are divisible by 5, so $f_{5n}$ is divisible by 5.

\section{8.2, 4c}


\section{8.2, 4e}

\section{8.2, 4f}

\section{8.2, 12}

\section{8.2, 26b}

\section{8.2, 26d}

\section{9.1, 6b}

\section{9.1, 6f}

\section{9.1, 32}

\section{9.1, 56b}

\end{document}