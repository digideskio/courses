\documentclass{article}
\usepackage[utf8]{inputenc}
\usepackage[margin=1in]{geometry}

\title{Discrete Math: Problem Set #3}
\author{Jackie Luo}
\date{March 25, 2015}

\begin{document}

\maketitle

\section{3.2, 2a}
$f(x) = 17x + 11$
\newline
$0 \leq 17x + 11 \leq 17x + 11x = 28x$ whenever $x > 1$
\newline
$C = 28, k = 1$
\newline
$f(x) = 17x + 11 < 28x$ whenever $x > 1$
\newline
\newline
$f(x)$ is $O(n)$, not $O(n^2)$.

\section{3.2, 2c}
$f(x) = x \log x$
\newline
$0 \leq x \log x \leq x^2$ whenever $x > 1$
\newline
$C = 1, k = 1$
\newline
$f(x) = x \log x < x^2$ whenever $x > 1$
\newline
\newline
$f(x)$ is $O(n^2)$.

\section{3.2, 8d}
$f(x) = (x^3 + 5 \log x)/(x^4 + 1)$
\newline
$x^3 + 5 \log x$ is $O(x^3)$ because, at large numbers, $5 \log x$ has a negligible impact on the sum. In the denominator, $x^4 + 1$ is $O(x^4)$ because $1$ is a constant and has virtually no impact on the sum. With $O(x^3)$ and $O(x^4)$, $f(x)$ is $O(1/x)$, or $O(x^{-1})$.

\section{3.2}
\subsection{30a}
$f(x) = 3x + 7$
\newline
$g(x) = x$
\newline
$x/2 \leq 3x + 7 \leq 10x$ whenever $x > 1$
\newline
The existence of $C_{1} = 1/2$, $C_{2} = 10$, and $k = 1$ means that $f(x) = \theta (g(x))$.

\subsection{30b}
$f(x) = 2x^2 + x - 7$
\newline
$g(x) = x^2$
\newline
$x^2 \leq 2x^2 + x - 7 \leq 3x^2$ whenever $x > 2$
\newline
The existence of $C_{1} = 1$, $C_{2} = 3$, and $k = 2$ means that $f(x) = \theta (g(x))$.

\subsection{30c}
$f(x) = \left \lfloor{x + 1/2}\right \rfloor$
\newline
$g(x) = x$
\newline
$x/2 \leq \left \lfloor{x + 1/2}\right \rfloor \leq 2x$ whenever $x > 1$
\newline
The existence of $C_{1} = 1/2$, $C_{2} = 2$, and $k = 1$ means that $f(x) = \theta (g(x))$.

\subsection{30d}
$f(x) = \log{x^2 + 1}$
\newline
$g(x) = \log_2 x$
\newline
$1/2(\log_2 x) \leq \log{x^2 + 1} \leq 5\log_2$ whenever $x > 1$
\newline
The existence of $C_{1} = 1/2$, $C_{2} = 5$, and $k = 1$ means that $f(x) = \theta (g(x))$.

\subsection{30e}
$f(x) = \log_{10} x$
\newline
$g(x) = \log_2 x$
\newline
$1/4(\log_2 x) \leq \log_{10} \leq \log_2$ whenever $x > 1$
\newline
The existence of $C_{1} = 1/4$, $C_{2} = 1$, and $k = 1$ means that $f(x) = \theta (g(x))$.

\section{3.1, 32}
sum = 0
\newline
list = []
\newline
for number in sequence:
\newline
\indent if number $>$ sum:
\newline
\indent \indent list.append(number)
\newline
\indent sum $+=$ number
\newline
return list

\section{3.3, 30}
The worst-case time complexity of the algorithm is $O(n)$ because it goes through the sequences of numbers once to get all of the desired terms.

\section{4.2, 30c}
$37 = 1(3^3) + 1(3^2) + 0(3^1) + 1(3^0) = 1(27) + 1(9) + 0(3) + 1(1) = 27 + 9 + 1$

\section{4.3, 28}
Using prime factorization, we can find gcd(1000, 625) and lcm(1000, 625):
\newline
$1000 = 2^3 \times 5^3$
\newline
$625 = 5^4$
\newline
gcd(1000, 625) = $5^3$ = 125
\newline
lcm(1000, 625) = $2^3 \times 5^4$ = 5000
\newline
$1000 \times 625 = 125 \times 5000 = 625000$

\section{4.3, 32e}
gcd(1529, 14038)
\newline
A = 14038, B = 1529
\newline
$14038/1529 = 9$ mod $277$
\newline
$1529/277 = 5$ mod $144$
\newline
$277/144 = 1$ mod $133$
\newline
$144/133 = 1$ mod $11$
\newline
$133/11 = 12$ mod $1$
\newline
$11/1 = 11$ mod $0$
\newline
gcd(1529, 14038) = 11

\section{4.3, 40c}
gcd(35, 78)
\newline
$78 = 35 \times 2 + 8$
\newline
$35 = 8 \times 4 + 3$
\newline
$8 = 3 \times 2 + 2$
\newline
$3 = 2 \times 1 + 1$
\newline
$2 = 1 \times 2$
\newline
gcd(35, 78) = 1
\newline
$= 3 - 2 \times 1$
\newline
$= 3 - (8 - 3 \times 2) \times 1 = 3 \times 3 - 8$
\newline
$= (35 - 8 \times 4) \times 3 - 8 = 35 \times 3 - 8 \times 11$
\newline
$= 35 \times 3 - (78 - 35 \times 2) \times 11 = -78 \times 11 + 35 \times 25$

\section{4.3, 44}
gcd(1001, 100001)
\newline
$100001 = 1001 \times 99 + 902$
\newline
$1001 = 902 \times 1 + 99$
\newline
$902 = 99 \times 9 + 11$
\newline
$99 = 11 \times 9$
\newline
gcd(1001, 100001) = 11
\newline
$= 902 - 99 \times 9$
\newline
$= 902 - (1001 - 902 \times 1) \times 9 = -1001 \times 9 +902 \times 10$
\newline
$= -1001 \times 9 + (100001 - 1001 \times 99) \times 10 = 100001 \times 10 - 1001 \times 999$

\section{4.3, 50}
$m$ must divide $a - b$, as we've seen in the Euclidean algorithm. As a result, we know that gcd(a, m) divides $a - b$. Because gcd(a, m) also divides $a$, it must divide $b$. Since it divides $m$, gcd(a, m) divides gcd(b, m). We can go through the same logic with gcd(b, m), meaning that the two divide each other and therefore are equal.

\section{4.4, 6b}
a = 34, m = 89
\newline
$89 = 34 \times 2 + 21$
\newline
$34 = 21 \times 1 + 13$
\newline
$21 = 13 \times 1 + 8$
\newline
$13 = 8 \times 1 + 5$
\newline
$8 = 5 \times 1 + 3$
\newline
$5 = 3 \times 1 + 2$
\newline
$3 = 2 \times 1 + 1$
\newline
$2 = 1 \times 2$
\newline
gcd(34, 89) = 1
\newline
$= 3 - 2 \times 1$
\newline
$= 3 - (5 - 3 \times 1) \times 1 = 3 \times 2 - 5$
\newline
$= (8 - 5 \times 1) \times 2 - 5 = 8 \times 2 - 5 \times 3$
\newline
$= 8 \times 2 - (13 - 8 \times 1) \times 3 = 8 \times 5 - 13 \times 3$
\newline
$= (21 - 13 \times 1) \times 5 - 13 \times 3 = 21 \times 5 - 13 \times 8$
\newline
$= 21 \times 5 - (34 - 21 \times 1) \times 8 = 21 \times 13 - 34 \times 8$
\newline
$= (89 - 34 \times 2) \times 13 - 34 \times 8 = 89 \times 13 - 34 \times 34$
\newline
13 is the inverse of 34 modulo 89.

\section{4.4, 34}
$23^{1002}$ mod $41$
\newline
$23^{40} = 1$ mod $41$, so $(23^{40})^k = 1$ mod $41$ for every positive integer $k$
\newline
$23^{1002} = 23^{40 \times 25 + 2} = (23^{40})^{25} \times 23^2 = 1^{25} \times 529 = 37$ (mod 41)

\section{4.6, 2b}
"STOP POLLUTION" = 18 19 14 15 // 15 14 11 11 20 19 8 14 13
\newline
Using $f(p) = (p + 21)$ mod $26$:
\newline
13 14 9 10 // 10 9 6 6 15 14 3 9 8 = "NOJK KJGGPODJI"

\section{4.6, 4a}
"EOXH MHDQV" = 4 14 23 7 // 12 7 3 16 21
\newline
Using $-k = -3$ mod $26$:
\newline
1 11 20 4 // 9 4 0 13 18 = "BLUE JEANS"

\section{4.6, 10}
Yes, you can use $k = 13$ as the enciphering function and deciphering function for a shift cipher. For example, for the example of "STOP POLLUTION," the original message is: 18 19 14 15 // 15 14 11 11 20 19 8 14 13. If we add 13 to each number and modulo 26, we get 5 6 1 2 // 2 1 24 24 7 6 21 1 0. Using that same function ($f(p) = (p + 13)$ mod $26$), we get the original message, 18 19 14 15 // 15 14 11 11 20 19 8 14 13.

\section{4.6, 24}
"ATTACK" = 0120 2001 0311
\newline
$C = M^{13}$ mod $2537$
\newline
$C = 120^{13}$ mod $43 = (-9)^{13}$ mod $43 = 28$
\newline
$C = 120^{13}$ mod $59 = 2^{13}$ mod $59 = 8192$ mod $59 = 50$
\newline
$120^{13} = 28 \times 59 \times 35 + 50 \times 43 \times 11 ($mod $2537) = 2006 + 817 ($mod $2537) = 286$
\newline
Computing the other two blocks, the message is 268 798 425.

\section{4.6, 26}
To find the inverse of 17 modulo $52 \times 60$:
\newline
$3120 = 17 \times 183 + 9$
\newline
$17 = 9 \times 1 + 8$
\newline
$9 = 8 \times 1 + 1$
\newline
$8 = 1 \times 8$
\newline
gcd(17, 3120) = 1
\newline
$= 9 - 8 \times 1$
\newline
$= 9 - (17 - 9 \times 1) \times 1 = 9 \times 2 - 17$
\newline
$= (3120 - 17 \times 183) \times 2 - 17 = 3120 \times 2 - 17 \times 367$
\newline
$-17 \times 367 = 1 - 3120 \times 2$
\newline
$-367 = 2753$ mod $3120$, so $d = 2753$
\newline
\newline
To find the original message:
\newline
$3185^{2753}$ mod $53 \times 61 = 1816$
\newline
$2038^{2753}$ mod $53 \times 61 = 2008$
\newline
$2460^{2753}$ mod $53 \times 61 = 1717$
\newline
$2550^{2753}$ mod $53 \times 61 = 411$
\newline
In other words, the message is 1816 2008 1717 0411, or "SQUIRREL."

\end{document}