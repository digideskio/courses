\documentclass{article}
\usepackage[utf8]{inputenc}
\usepackage[margin=1in]{geometry}
\usepackage{tikz}
\usepackage{amsmath}

\title{Discrete Math: Problem Set #2}
\author{Jackie Luo}
\date{February 25, 2015}

\begin{document}

\maketitle

\section{2.1, 10f}
$\{\{\emptyset\}\} \subset \{\emptyset, \{\emptyset\}\}$ is false.

\section{2.1, 22}
Yes, if $A$ and $B$ are two sets with the same power set, then $A = B$. If $A \neq B$, then they can't have the same power set because each would contain a subset not contained in the other. In conclusion, $A = B$ if and only if A and B have the same power set.

\section{2.2, 14}
A = $\{1, 3, 5, 6, 7, 8, 9\}$
\newline
B = $\{2, 3, 6, 9, 10\}$

\section{2.2, 26c}
$(A - B) \cup (A - C) \cup (B - C)$
\newline
\begin{center}
\def\firstcircle{(90:1.75cm) circle (2.5cm)}
\def\secondcircle{(210:1.75cm) circle (2.5cm)}
\def\thirdcircle{(330:1.75cm) circle (2.5cm)}
\begin{tikzpicture}
    \draw \firstcircle node[text=black,above] {$A$};
    \draw \secondcircle node [text=black,below left] {$B$};
    \draw \thirdcircle node [text=black,below right] {$C$};
\end{tikzpicture}
\end{center}

\section{2.2, 30c}
Yes, $A = B$ if $A \cup C = B \cup C$ and $A \cap C = B \cap C$.

\section{2.3, 22c}
No, $f(x) = (x + 1)/(x + 2)$ is not a bijection from R to R because it's undefined with a vertical asymptote at $x = 2$.

\section{2.3, 38}
$f \enspace o \enspace g = g \enspace o \enspace f
\newline
a(cx + d) + b = c(ax + b) + d
\newline
acx + ad + b = acx + bc + d
\newline
\textit{\textbf{ad + b = bc + d}}
\newline
\textit{\textbf{a = (bc - b + d)/d}}
\newline
b - bc = d - ad
\newline
b(1 - c) = d - ad
\newline
\textit{\textbf{b = (d - ad)/(1 - c)}}
\newline
bc = ad + b - d
\newline
\textit{\textbf{c = (ad + b - d)/b}}
\newline
ad - d = bc - b
\newline
d(a - 1) = bc - b
\newline
\textit{\textbf{d = (bc - b)/(a - 1)}}$

\section{2.4, 26e}
$a_n = 3^{n-1} - 1$
The next three terms are 59,048, 177,146, and 531,440.

\section{2.4, 32b}
$\sum\limits_{i=0}^8 (3^i - 2^i)$
\newline
$ = (3^0 - 2^0) + (3^1 - 2^1) + (3^2 - 2^2) + (3^3 - 2^3) + (3^4 - 2^4) + (3^5 - 2^5) + (3^6 - 2^6) + (3^7 - 2^7) + (3^8 - 2^8)$
\newline
$ = (1 - 1) + (3 - 2) + (9 - 4) + (27 - 8) + (81 + 16) + (243 - 32) + (729 - 64) + (2187 - 128) + (6561 - 256)$
\newline
$ = 0 + 1 + 5 + 19 + 65 + 211 + 665 + 2059 + 6305$
\newline
$ = 9330$

\section{2.4, 34b}
$\sum\limits_{i=0}^3 \sum\limits_{j=0}^2 (3i + 2j)$
\newline
$\sum\limits_{j=0}^2 (3i + 2j) = \sum\limits_{j=0}^2 3i + \sum\limits_{j=0}^2 2j = 3i \sum\limits_{j=0}^2 1 + 2 \sum\limits_{j=0}^2 j = 3i(3) + 2(0 + 1 + 2) = 9i + 6$
\newline
$\sum\limits_{i=0}^3(\sum\limits_{j=0}^2 (3i + 2j)) = \sum\limits_{i=0}^3 (9i + 6) = \sum\limits_{i=0}^3 9i + \sum\limits_{i=0}^3 6 = 9 \sum\limits_{i=0}^3 i + 6 \sum\limits_{i=0}^3 1 = 9(0 + 1 + 2 + 3) + 6(4) = 54 + 24 = 78$

\section{5.1, 6}
$1 \times 1! + 2 \times 2! + ... + n \times n! + (n + 1) \times (n + 1)!
\newline
= (1 \times 1! + 2 \times 2! + ... + n \times n!) + (n + 1) \times (n + 1)!
\newline
= (n + 1)! - 1 + (n + 1) \times (n + 1)!
\newline
= (n + 2)! - 1$
\newline
The proof by induction shows that $1 \times 1! + 2 \times 2! + ... + n \times n! = (n + 1)! - 1$ whenever $n$ is a positive integer.

\section{5.1, 10}
\subsection{a}
The first five values of the sequence are 1/2, 2/3, 3/4, 4/5, and 5/6. The formula for the sequence, then, is n/(n + 1).
\subsection{b}
$1/(1 \times 2) + 2/(2 \times 3) + ... + 1/n(n + 1) + 1/(n + 1)(n + 1 + 1)
\newline
= (1/(1 \times 2) + 2/(2 \times 3) + ... + 1/n(n + 1)) + 1/(n + 1)(n + 1 + 1)
\newline
= n/(n + 1) + 1/(n + 1)(n + 2)
\newline
= n(n + 2)/(n + 1)(n + 2) + 1/(n + 1)(n + 2)
\newline
= (n^2 + 2n + 1)/(n + 1)(n + 2)
\newline
= (n + 1)^2/(n + 1)(n + 2)
\newline
= (n + 1)/(n + 2)$
\newline
The proof by induction shows that $1/(1 \times 2) + 2/(2 \times 3) + ... + 1/n(n + 1) = n/(n + 1)$.

\section{5.1, 16}
$1 \times 2 \times 3 + 2 \times 3 \times 4 + ... + n(n + 1)(n + 2) + (n + 1)(n + 2)(n + 3)
\newline
= (1 \times 2 \times 3 + 2 \times 3 \times 4 + ... + n(n + 1)(n + 2)) + (n + 1)(n + 2)(n + 3)
\newline
= n(n + 1)(n + 2)(n + 3)/4 + (n + 1)(n + 2)(n + 3)
\newline
= n(n + 1)(n + 2)(n + 3)/4 + 4(n + 1)(n + 2)(n + 3)/4
\newline
= (n(n + 1)(n + 2)(n + 3) + 4(n + 1)(n + 2)(n + 3))/4
\newline
= ((n + 1)(n + 2)(n + 3)(n + 4)/4$
\newline
The proof by induction shows that $1 \times 2 \times 3 + 2 \times 3 \times 4 + ... + n(n + 1)(n + 2) = n(n + 1)(n + 2)(n + 3)/4$ for every positive integer $n$.

\section{5.1, 32}
Basis Step: $1 + 2 = 3$, which is divisible by 3.
\newline
Inductive Step:
\newline
$(n + 1)^3 + 2(n + 1)
\newline
= n^3 + 3n^2 + 3n + 1 + 2n + 2
\newline
= (n^3 + 2n) + 3n^2 + 3n + 3
\newline
= (n^3 + 2n) + 3(n^2 + n + 1)$
\newline
The first term, $n^3 + 2n$, is divisible by 3 by the theorem in Section 4.1, and the second term, $3(n^2 + n + 1)$ must be divisible by 3 as well. Thus, $(n^3 + 2n) + 3(n^2 + n + 1) = (n + 1)^3 + 2(n + 1)$ is divisible by 3 whenever $n$ is a positive integer.

\newpage
\section{5.2, 12}
Let $P(n)$ be the proposition that $n$ can be written as a sum of distinct powers of two.
\newline
Basis Step: $2^0 = 1$, so $P(1)$ is true.
\newline
Induction Step (EVEN):
\newline
If $n + 1$ is even, then $(n + 1)/2$ is an integer.
\newline
$(n + 1)/2 = 2^{a1} + 2^{a2} + ... + 2^{ak}
\newline
n + 1 = 2(2^{a1} + 2^{a2} + ... + 2^{ak})
\newline
n + 1 = 2^{a1 + 1} + 2^{a2 + 1} + ... + 2^{ak + 1}$
\newline
Induction Step (ODD):
\newline
Because $n + 1$ is odd, $k$ must be even, which means that none of the distinct powers of two can be $2^0$, or 1. Even from the previous inductive step, we know that any even $n + 1$ can be written as a sum of distinct powers of two, so if $2^0$ is necessarily excluded from all even integers, then we know that all odd integers can be written as a sum of distinct powers of two as well.
\newline
$n + 1 = 2^0 + 2^{a1} + 2^{a2} + ... + 2^{ak}$

\section{5.2, 14}
Basis Step: In a pile with two stones, the split gives you two piles with one stone in each pile. $2(2-1)/2 = 1$, so $n(n-1)/2$ holds true for the basis step.
\newline
Induction Step:
\newline
If we begin with $n$ stones, they can be split into $r$ and $s$ stones. $s = n - r$, so we have $r(r - 1)/2$ and $(n - r)(n - r - 1)/2$. The product of the stones is $r(n - r)$.
\newline
$r(r - 1)/2 + (n - r)(n - r - 1)/2 + r(n - r)
\newline
= r(r - 1)/2 + (n - r)(n - r - 1)/2 + 2r(n - r)/2
\newline
= (r^2 - r + n^2 - 2nr - n + r^2 + r + 2nr - 2r^2)/2
\newline
= (n^2 - n)/2
\newline
= n(n - 1)/2$
\newline
By strong induction, we can tell that the sum of the products computed at each step equals $n(n - 1)/2$.

\section{5.2, 30}
The flaw in the proof is the basis step. The inductive step calls for $a^{k}$ and $a^{k - 1}$, and we only have basis for $a^{k}$. $a^{-1} \neq 1$, which means that the inductive step isn't supported.

\section{5.3, 8d}
$a_{1} = 1^2 = 1
\newline
a_{n} = a_{n - 1} + 2n - 1$ for all $n > 1$

\section{5.3, 24}
\subsection{a}
$a_{1} = 1
\newline
a_{n} = a_{n - 1} + 2$ for all $n > 1$
\subsection{b}
$a_{1} = 3^1 = 3
\newline
a_{n} = a_{n - 1} \times 3$ for all $n > 1$

\section{5.3, 28c}


\end{document}