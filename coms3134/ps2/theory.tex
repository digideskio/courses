\documentclass{article}
\usepackage[utf8]{inputenc}
\usepackage[margin=1in]{geometry}
\usepackage{tikz}
\usetikzlibrary{trees}

\title{Data Structures: Problem Set #2}
\author{Jackie Luo}
\date{February 27, 2015}

\begin{document}

\maketitle

\section{Theory}

\subsection{}
a. For the specific input train, the steps are as follows:
\newline
\newline
1. Move Car 3 to top of $S_{1}$.
\newline
2. Move Car 6 to top of $S_{2}$.
\newline
3. Move Car 9 to top of $S_{3}$.
\newline
4. Move Car 2 to top of $S_{1}$.
\newline
5. Move Car 4 to top of $S_{2}$.
\newline
6. Move Car 7 to top of $S_{3}$.
\newline
7. Move Car 1 to back of output track.
\newline
8. Move Car 2 from top of $S_{1}$ to back of output track.
\newline
9. Move Car 3 from top of $S_{1}$ to back of output track.
\newline
10. Move Car 4 from top of $S_{2}$ to back of output track.
\newline
11. Move Car 8 to top of $S_{1}$.
\newline
12. Move Car 5 to back of output track.
\newline
13. Move Car 6 from top of $S_{2}$ to back of output track.
\newline
14. Move Car 7 from top of $S_{3}$ to back of output track.
\newline
15. Move Car 8 from top of $S_{1}$ to back of output track.
\newline
16. Move Car 9 from top of $S_{3}$ to back of output track.
\newline
\newline
b. No, there is not a solution for any train of length 9 with 3 holding tracks because the three operations allowed (front of input to back of output, front of input to top of holding, and top of holding to back of output) can limit the organization of the cars. For instance, if the cars were ordered [1, 9, 8, 7, 6, 5, 4, 3, 2], they could not be arranged in increasing order because you wouldn't be able to place the cars in the holding tracks in decreasing order (smallest numbers on top, largest on the bottom), and you would need to hold all of the cars before moving Car 1 to the back of the output track.
\newline
c. If the car at the front of the input track is the next in order for the output track, move it to the back of the output track. Otherwise, place it in a holding track. Choose the holding track with the car with the closest number larger than the current car. If all tracks are empty or there isn't a holding track with a car with a larger number, then place the car in an empty track. Whenever their numbers in the sequence are reached, move cars from the top of the holding tracks to the back of the output track.

\newpage
\subsection{}
Basis Step: $2^{h + 1} - 1 = 2^{1 + 1} - 1 = 2^2 - 1 = 4 - 1 = 3$
\newline
The basis step holds true because a binary tree of height 1 (one edge length between nodes) can contain a maximum of 3 nodes.
\newline
Inductive Step:
\newline
$2^{k + 2} - 1
\newline
= 2^0 + 2^1 + 2^2 + ... + 2^k + 2^{k + 1}
\newline
= (2^{k + 1} - 1) + 2^{k + 1}
\newline
= 2(2^{k + 1}) - 1
\newline
= 2^{k + 2} - 1$
\newline
The inductive step proves that $2^{h + 1} - 1$ is the maximum number of nodes in a binary tree of height $h$.

\subsection{}
If we have a full binary tree and only know the value of $L$, then the total number of nodes is $L + (L - 1)$, or $2L - 1$. We know that all leaves are nodes, so $L$ is automatically included in the total number of nodes. The number of internal nodes is always $L - 1$. For instance, in a full binary tree with three nodes (two leaves, one internal node), when you add two nodes to one of the leaves, that results in one more leaf (two added, one no longer a leaf) and one more internal node, which shows that the difference will always stay constant.

\subsection{}
a. Starting from the root, any value larger goes to the right, and any value lower goes to the left.
\newline
\newline
\begin{tikzpicture}[level/.style={sibling distance=40mm/#1}]
    \node [circle,draw] {3}
      child {
        node [circle,draw] {2}
            child {node [circle,draw] {1}}
        child [missing]}
      child {
        node [circle,draw] {4}
        child [missing]
        child {node [circle,draw]  {5}
        child [missing]
        child {node [circle, draw]  {9}
        child {node [circle, draw]  {6}}
        child [missing]}}
    };
\end{tikzpicture}
\newpage
b. When the root is deleted, we traverse down the left-hand side of the binary search tree for the highest value (in this case, 2).
\newline
\newline
\begin{tikzpicture}[level/.style={sibling distance=40mm/#1}]
    \node [circle,draw] {2}
      child {
        node [circle,draw] {1}}
      child {
        node [circle,draw] {4}
        child [missing]
        child {node [circle,draw]  {5}
        child [missing]
        child {node [circle, draw]  {9}
        child {node [circle, draw]  {6}}
        child [missing]}}
    };
\end{tikzpicture}

\end{document}